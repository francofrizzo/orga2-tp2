\section{Experimentación}
    En esta sección, se presentan las pruebas experimentales realizadas sobre ambas implementaciones de los algoritmos, como así también los resultados obtenidos en cada una de ellas y una discusión de los mismos. El objetivo detrás de la realización de estos experimentos fue evaluar y comparar el rendimiento de las diferentes implementaciones, para extraer conclusiones acerca de las características del modelo de programación \acr{SIMD}.

    Todas las pruebas pueden ser recreadas mediante la ejecución de una serie de scripts incluidos con los archivos fuentes del trabajo práctico, como así también los gráficos que se incluyen en este informe. Estos scripts, programados en Bash, se encuentran en el directorio \texttt{exp}, y llevan por nombre \texttt{exp{i}}, donde \texttt{i} es el número del experimento. El parámetro opcional \texttt{-n <cant>} permite elegir la cantidad de veces que se rtepetirá cada medición.

    \subsection{Experimento 1: Comparación de rendimiento}

        \subsubsection*{Presentación}
        	En este experimento, se buscó comparar el rendimiento de las implementaciones en lenguaje C de ambos filtros, con el obtenido con las respectivas implementaciones en lenguaje ensamblador utilizando el paradigma \acr{SIMD}. Las pruebas se realizaron con diferentes tamaños de imágenes, con el objetivo de lograr una comparación independiente de dicho parámetro y, al mismo tiempo, adquirir una idea del comportamiento empírico de cada implementación con respecto al tamaño de la entrada.

        \subsubsection*{Metodología, datos y parámetros del experimento}
        	Todas las instancias de experimentación se ejecutaron 20 veces, calculando luego el promedio de los valores obtenidos
        	En el primer experimento se genera una serie de imágenes de diferentes tamaños, tomando una imagen grande y disminuyendo progresivamente sus dimensiones.
        	Luego, se ejecuta el filtro \emph{blur} con cada una de las imágenes generadas y se compara el tiempo de ejecución de las implementaciones en C y lenguaje ensamblador.
        	Esto se repite para el filtro \emph{diff}, con la diferencia de que para cada tamaño de imagen se genera 
        	un par de imágenes con ciertas diferencias entre ellas, para poder verificar el buen funcionamiento del mismo.

        \subsubsection*{Hipótesis}

        \subsubsection*{Resultados obtenidos y discusión}

    \subsection{Experimento 2: Rendimiento según parámetros del filtro}

        \subsubsection*{Presentación}

        \subsubsection*{Metodología, datos y parámetros del experimento}

        \subsubsection*{Hipótesis}

        \subsubsection*{Resultados obtenidos y discusión}

    \subsection{Experimento 3: Peso de llamados a función}

        \subsubsection*{Presentación}

        \subsubsection*{Metodología, datos y parámetros del experimento}

        \subsubsection*{Hipótesis}

        \subsubsection*{Resultados obtenidos y discusión}
