\section{Introducción}

  En el presente trabajo, se aplica el modelo de programación vectorial \acr{SIMD} (\emph{Single Instruction, Multiple Data}) para la implementación de filtros para el procesamiento de imágenes. Más precisamente, se lleva a cabo la implementación de los siguientes dos filtros:
  \begin{itemize}
    \item \emph{Diferencia} (\emph{diff}), que recibe como entrada dos imágenes y devuelve como resultado otra imagen que indica dónde difieren las dos primeras.
    \item \emph{Blur gaussiano} (\emph{blur}), que suaviza la imagen reemplazando cada píxel por un promedio de los píxeles circundantes, ponderado según una función gaussiana.
  \end{itemize}

  La elaboración del trabajo se dividió en dos etapas. En primer lugar, se implementaron ambos filtros tanto en lenguaje C como en lenguaje ensamblador para la arquitectura x86 de Intel. En este último caso, se utilizaron las instrucciones \acr{SSE} de dicha arquitectura, que aprovechan el ya mencionado modelo \acr{SIMD} para procesar datos en forma paralela.

  Una vez realizadas estas implementaciones, fueron sometidas a un proceso de comparación para extraer conclusiones acerca de su rendimiento. Con este fin, se experimentó con variaciones tanto en los datos de entrada como en detalles implementativos de los mismos algoritmos. De esta manera, se pudo recopilar datos sobre el comportamiento de cada implementación, y contrastar estos resultados con diversas hipótesis previamente elaboradas.
